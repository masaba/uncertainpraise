% Template for Cogsci submission with R Markdown

% Stuff changed from original Markdown PLOS Template
\documentclass[10pt, letterpaper]{article}

\usepackage{cogsci}
\usepackage{pslatex}
\usepackage{float}
\usepackage{caption}

% amsmath package, useful for mathematical formulas
\usepackage{amsmath}

% amssymb package, useful for mathematical symbols
\usepackage{amssymb}

% hyperref package, useful for hyperlinks
\usepackage{hyperref}

% graphicx package, useful for including eps and pdf graphics
% include graphics with the command \includegraphics
\usepackage{graphicx}

% Sweave(-like)
\usepackage{fancyvrb}
\DefineVerbatimEnvironment{Sinput}{Verbatim}{fontshape=sl}
\DefineVerbatimEnvironment{Soutput}{Verbatim}{}
\DefineVerbatimEnvironment{Scode}{Verbatim}{fontshape=sl}
\newenvironment{Schunk}{}{}
\DefineVerbatimEnvironment{Code}{Verbatim}{}
\DefineVerbatimEnvironment{CodeInput}{Verbatim}{fontshape=sl}
\DefineVerbatimEnvironment{CodeOutput}{Verbatim}{}
\newenvironment{CodeChunk}{}{}

% cite package, to clean up citations in the main text. Do not remove.
\usepackage{cite}

\usepackage{color}

% Use doublespacing - comment out for single spacing
%\usepackage{setspace}
%\doublespacing


% % Text layout
% \topmargin 0.0cm
% \oddsidemargin 0.5cm
% \evensidemargin 0.5cm
% \textwidth 16cm
% \textheight 21cm

\title{Young children use statistical evidence to infer the informativeness of
praise}


\author{{\large \bf Mika Asaba},  {\large \bf Emily Hembacher}, {\large \bf Helen Qiu}, {\large \bf Brett Anderson}, {\large \bf Michael Frank}, {\large \bf Hyowon Gweon} \\ \{masaba, ehembach, shiqiu12, brettand, mcfrank, hyo\} @stanford.edu \\ Department of Psychology, Stanford University}

\begin{document}

\maketitle

\begin{abstract}
Receiving praise is not only rewarding but also informative for
learners. It allows them to learn about their skills and competence even
when they are uncertain or unable to judge for themselves. Not all
praise is equally meaningful, however: praise from someone who praises
indiscriminately is less informative than from someone who praises
selectively. Do young children understand whose praise is more
informative? Here we ask whether young children use covariation
information to infer the informativeness of others' praise. Adults (Exp.
1) and 4-5 year-olds (Exp. 2) were more likely to trust the praise from
a selective teacher whose previous praise covaried with the quality of
work than praise from a teacher who indiscriminately praised independent
of its quality. Exp. 3 (4 year-olds) addressed the possibility that
children simply prefer a teacher who praises less often. Even for young
children, praise is more than something nice; they track the
informativeness of others' evaluative feedback and use it to learn about
the quality of their own work.

\textbf{Keywords:}
praise, theory of mind, uncertainty
\end{abstract}

Evaluative feedback from others is an important source of information
for learning about ourselves. In the face of uncertainty about our
abilities, traits, or the quality of our work, feedback from others can
serve as a useful indicator of quality or success. For instance, if you
just gave a big presentation at work and are unsure of how it went,
receiving praise from a colleague (e.g., ``that was a great talk!'')
could give you more certainty that it was good.

Not all praise is equally meaningful, however. We often interpret
others' evaluative feedback in the context of the evaluator, considering
her expertise, personality, or communicative goals. For instance, if you
knew that your colleague only praises the best talks, you would be
thrilled to have received praise; if your colleague is known to be
overgenerous with praising talks, you might remain uncertain about how
well you really did. The informativeness of positive evaluations can be
especially ambiguous, because others may be driven not only by the goal
to provide an accurate evaluation (an epistemic goal) but also by the
desire to make others feel good (a social goal). Recent work suggests
that adults are indeed sensitive to both the epistemic and social
utility of evaluative feedback and interpret it based on speakers'
communicative goals (Yoon, Tessler, Goodman, \& Frank, 2016).

Praise is considered to be a useful tool to foster motivation and
achievement and is common in parenting and educational practices in many
societies. Because children routinely receive praise from others,
understanding whose praise is more informative can help them learn
better about their own skills and abilities. Despite much work on the
effect of praise on children's intrinsic motivation (e.g., Henderlong \&
Lepper (2002)), we understand little about \textit{how} children
interpret the meaning of praise and the cognitive capacities that
underlie this ability. As a first step to address these questions, we
investigate whether adults (Exp. 1) and young children (4-5 year-olds;
Exp. 2-3) track the informativeness of others' praise and use it to
evaluate the quality of their own work. As the meaning of evaluative
feedback often depends on the evaluator, in this initial evaluation we
use participants' endorsement of praise as a proxy for their evaluation
of the praiser.

Understanding the informativeness of praise might be challenging for
young children, especially if they have difficulty differentiating
informativeness and niceness. Prior work suggests that young children
attribute knowledge to nice informants and endorse novel information
from them even when they lack perceptual access (Lane, Wellman, \&
Gelman, 2013). Even school-aged children show a tendency to endorse
positive assessments of an agent (Boseovski, 2012) or that agent's
drawing or painting, despite negative evaluations from multiple others
(Boseovski, Marble, \& Hughes, 2017). In these studies however, children
did not have access to any information about the work being evaluated
aside from the informants' praise. In the absence of a means to evaluate
the accuracy of the testimony, children might have relied more on
informants' niceness. To the extent that children have a clear
understanding of the actual quality of the work being evaluated, they
may be sensitive to whether informants provide evaluations that are
consistent with the quality of the work. \%, and use this to learn about
their informativeness.

Support for this hypothesis comes from related work in epistemic trust,
which suggests that 3-5 year-old children readily track the
informativeness of others when a clear ground truth is available to
them. Children preferentially learn from teachers who previously
provided correct (versus incorrect) labels of familiar objects (e.g.,
Koenig \& Harris (2005); see Sobel \& Kushnir (2013) for a review), and
update their evaluations of others' trustworthiness across multiple
interactions (Ronfard \& Lane, 2017). Beyond tracking inaccuracies,
children also recognize more subtle forms of misinformation; when a
teacher demonstrates only one of four functions of a toy, children
penalize these teachers (Gweon \& Asaba, 2017) and are less likely to
trust the teacher in learning about a new toy (Gweon, Pelton, Konopka,
\& Schulz, 2014). These results suggest that preschool-aged children do
not simply accept information from others as true of the world. Instead,
they monitor others' informativeness based on the quality of their
testimony or demonstrations and selectively endorse subsequent
information from them. Thus, children might demonstrate similar
sensitivity to the informativeness of others' praise.

Prior work also suggests that praise fosters intrinsic motivation for
learning and performance when it is directed at children's effort or the
product of their work, but not when it is directed at their intelligence
or ability (Henderlong Corpus \& Lepper, 2007; Mueller \& Dweck, 1998).
The fact that children respond differently to different kinds of praise
suggests that they understand praise as more than positive reinforcement
(Delin \& Baumeister, 1994). Indeed, researchers have proposed that the
effect of praise may depend on its perceived \textit{sincerity};
children may devalue its meaning when they see it as inflated,
unjustified, or inconsistent with their self-views or reality (Brophy,
1981; Delin \& Baumeister, 1994; Kanouse, Gumpert, \& Canavan-Gumpert,
1981). Consistent with this idea, recent work finds that inflated praise
has adverse effects on children with low self-esteem (Brummelman,
Thomaes, Orobio de Castro, Overbeek, \& Bushman, 2014). Yet, little
empirical work has directly investigated \textit{how} children recognize
the sincerity or informativeness of praise.

One possibility is that children learn others' informativeness from
repeated observations of the evaluative feedback they provide. Based on
whether their evaluation appropriately covaries with the actual quality
of the work (e.g., praise for high-quality work but not for low-quality
work) rather than independently of its quality (e.g., praise for both
high- and low-quality work), children may quickly form a model of the
evaluator even from minimal evidence. Informed by prior work on
children's ability to use minimal covariation data to make causal
attributions about themselves or others' behaviors (Gweon \& Schulz,
2011; Seiver, Gopnik, \& Goodman, 2013) and children's sensitivity to
teacher informativeness in epistemic domains (Bonawitz \& Shafto, 2016;
Gweon \& Asaba, 2017; Sobel \& Kushnir, 2013), we test the hypothesis
that children infer others' informativeness based on statistical
evidence, and use it to inform their evaluations of their work.

To this end, we chose an activity with which preschool-aged children are
familiar and motivated to improve on: tracing shapes with a marker. In
Experiment 1, we verify our task by showing that adult participants can
differentiate between a teacher who praises all tracings (Overpraise
Teacher) and a teacher who only praises tracings that are objectively
better (Selective Teacher), and they use this to appropriately evaluate
the quality of two hidden tracings. In Experiment 2, we designed a
similar, first-person paradigm and find that 4- and 5-year-olds can
differentiate between these two types of teachers and endorse the
Selective Teacher's praise. In Experiment 3, we further show that
children do not simply prefer teachers who provide praise less
frequently.

\section{Experiment 1: Adults}\label{experiment-1-adults}

We first verified that adults can track agents' patterns of praise and
infer their informativeness. Adults saw two teachers: a teacher whose
previous pattern of praise appropriately co-varied with the actual
quality of the tracings (Selective Teacher) and a teacher who previously
praised all tracings independently of their quality (Overpraise
Teacher). Given two unseen tracings, each endorsed by different
teachers, we predicted that adults would judge the one endorsed by the
Selective Teacher to be better in quality than the one endorsed by the
Overpraise Teacher. As an exploratory measure, we asked which teacher
was ``trying to be nice'' to see whether adults would distinguish the
teacher's goal to be informative from her goal to be nice or polite
(Yoon et al., 2016).

\subsection{Methods}\label{methods}

\subsubsection{Participants}\label{participants}

86 adults (38 female, \(M_{Age}\)(SD) = 36.5 (11.9), range = 21 - 71)
were recruited from Amazon's Mechanical Turk. An additional 14 subjects
were recruited but excluded because they failed one or both memory check
questions.

\subsubsection{Stimuli}\label{stimuli}

Images of ``good'' and ``bad'' tracings (i.e., a marker tracing that was
reasonably aligned or misaligned with the template shape, see Fig. 1A)
and two videos of teacher-student interactions were used (Fig. 1B). Both
videos showed a child (``Johnny'') with his back facing the camera, and
a teacher sitting across a table facing the child. Six tracings were
placed in a row on the table; there were 3 good tracings and 3 bad
tracings, in alternating order (similar in quality as those used in the
\textit{Warm-up Phase}). Each video had a different set of six tracings.
The same child appeared in both videos, but one featured ``Teacher
Jane'' who wore a green shirt and the other featured ``Teacher Susan''
who wore a red shirt.

\subsubsection{Procedure}\label{procedure}

In the \textit{Warm-up Phase}, participants were shown two tracings that
clearly differed in quality (Fig. 1A) and were asked to indicate which
one was better; participants who answered incorrectly were excluded from
analyses.

In the \textit{Teacher Introduction Phase}, participants watched two
videos. In both videos, Johnny first told the teacher that he made the
tracings and really wanted to know which of his tracings were good. The
teacher then evaluated the tracings one at a time from right to left. In
the Overpraise Teacher video, the teacher provided positive,
undifferentiated feedback (``Wow, that's great!'') for all six tracings
and placed a colored star sticker on each tracing that matched the color
of her shirt (see Fig. 1B). In the Selective Teacher video, the teacher
provided positive feedback on the good tracings (``Wow, that's great!'')
and put matching-colored stickers on them, while giving neutral feedback
(``Hm, this one's okay!'') for the three bad tracings without giving
stickers. Both teachers maintained a positive tone for both types of
feedback; teacher identity, pattern of praise, and order of presentation
were all counterbalanced. After each video, participants were asked how
many tracings the teacher said were great. Subjects who failed to
correctly answer these memory check questions (``3'' for the Selective
Teacher and ``6'' for the Overpraise Teacher) were excluded from
analyses.

Finally, in the \textit{Test Question Phase}, participants were shown a
picture of another student (Kristen) who made two tracings and two
envelopes, each of which contained one of her tracings. Participants
were told that the Selective Teacher saw only the tracing in one of the
envelopes and praised it (e.g., ``Teacher Jane said this tracing is
great''), and the Overpraise Teacher saw only the tracing in the other
envelope and praised it (``Teacher Susan said this tracing is great'').
Participants never saw Kristen's actual tracings, only the envelopes
with stickers that indicated which teacher praised the tracing.
Participants were asked: ``Kristen is going to bring one of her tracings
to a contest. Which tracing should she bring?'' Additionally,
participants were asked:''One of the teachers wanted to be nice. Who was
trying to be nice?"

\subsection{Results and Discussion}\label{results-and-discussion}

Our primary question was whether adults use teachers' prior patterns of
praise to evaluate the informativeness of subsequent praise. As
predicted, participants overwhelmingly chose the tracing praised by the
Selective Teacher (87.2\%, \(p\) \textless{} .001, Binomial Test, Figure
1D). Additionally, adults inferred niceness as the Overpraise Teacher's
communicative goal; a majority of participants chose the Overpraise
Teacher as the one who was \emph{trying} to be nice
(\(93.0\%, p < 0.001\), Binomial Test). \%These results confirm our
prediction that adults are sensitive to the informativeness of others'
feedback: These results suggest that adults readily detect the
differences in the informativeness of evaluative feedback from minimal
covariation data, and use it to inform their decisions about the quality
of an unseen product.

\section{Experiment 2}\label{experiment-2}

In Experiment 2, we investigated whether 4- and 5-year-old children are
also sensitive to the informativeness of others' praise. The design was
almost identical to Experiment 1, except that to increase engagement and
motivation, children themselves completed two tracings to be used in the
final Test Question.

If children can infer the informativeness of praise based on each
teacher's prior history of evaluating others' work, children would
choose the envelope praised by the Selective Teacher, as in Experiment
1. However, if children prefer to accept information from informants who
are perceived as positive or friendly, they would choose the tracing
praised by the Overpraise Teacher.

\subsection{Methods}\label{methods-1}

\subsubsection{Participants}\label{participants-1}

Forty four- and five-year-olds (19 female, \(M_{Age}\)(SD) = 4.91(0.42),
range = 4.1 - 5.91) were recruited from a local preschool\footnote{Planned
  sample size, exclusion criteria, and analysis plan preregistered at
  \url{https://aspredicted.org/4r9dh.pdf}}. An additional 5 subjects
were tested but excluded due to failure on the warm-up or memory
questions.

\subsubsection{Stimuli}\label{stimuli-1}

Stimuli were the same as those from Exp. 1. Additionally, two 8.5``x11''
tracing templates (a circle and either an overlapping triangle or
rectangle) were used for children to make their own tracings. The
tracings in the \textit{Warm-up Phase} were presented on laminated
sheets of paper, and videos were presented on a 13`` Macbook Pro laptop.
We also used printed pictures of the teachers and Johnny, two manila
envelopes, and star-shaped red and green stickers.

\subsubsection{Procedure}\label{procedure-1}

Children were tested in a private room in a preschool. During the
\textit{Warm-up Phase}, the experimenter first explained what tracing
is: ``The goal of tracing is to stay as close to the lines as possible''
and demonstrated tracing a rectangle for the child. Then, the child
traced the two templates, and the experimenter put each tracing away
into an envelope such that the child could not see the tracing for the
remainder of the session. Children were shown two pairs of tracings
(similar in quality to those in Exp. 1) and were asked to indicate which
one was better. Only children who passed both trials were included in
the analyses.

In the \textit{Teacher Introduction Phase}, children were shown a
picture of a student, Johnny. They were told that Johnny was working on
his tracings earlier and wanted help figuring out which of his tracings
were good, because he wanted to show them to his class later. Children
then watched the same Selective Teacher and Overpraise Teacher videos as
in Exp. 1. After each video, \%to verify that children understood and
remembered each teacher's pattern of praise, children saw a still frame
of the video (with no stars on the tracings) and were asked which
tracings the teacher said were great. \%Children responded by pointing
to the tracings. If children missed a tracing or incorrectly pointed to
a tracing that was not praised by the teacher, they watched the video
again and the experimenter asked the same memory check question. Those
who failed the memory check even after watching the video again were
excluded from analyses.

In the \textit{Test Question Phase}, the experimenter told the child
that Teacher Jane and Teacher Susan were nearby and could give feedback
on the child's tracings from earlier. The experimenter left the room
with the envelopes containing the child's tracings and returned after 15
seconds with stickers attached to the envelopes. The experimenter
pointed to the envelope with a green sticker and placed a photo of
Teacher Jane next to it, and said: ``Teacher Jane looked at this tracing
and said that this one is great.'' She then pointed to the other
envelope (with a red sticker and Teacher Susan's photo) and said:
``Teacher Susan looked at this tracing and said that this one is great''
(teacher order counterbalanced). Each envelope also had the
corresponding teacher's sticker on it.

Finally, with the tracings still in the envelopes, the experimenter
said: ``Now you can bring back your best tracing to show your teacher!
Which one do you think is the best?'' Children responded by pointing to
one of the envelopes. As an exploratory measure, children were asked,
``Which teacher was trying to be nice?''

\begin{CodeChunk}
\begin{figure*}[h]

{\centering \includegraphics{figs/methods-1} 

}

\caption[(A) Examples of good and bad tracings that subjects saw in the warm-up evaluation questions and teacher videos for Experiments 1-3]{(A) Examples of good and bad tracings that subjects saw in the warm-up evaluation questions and teacher videos for Experiments 1-3. (B) Teacher videos shown in Experiments 1-3. (C) Set-up and critical question for Experiments 2-3.}\label{fig:methods}
\end{figure*}
\end{CodeChunk}

\subsection{Results and Discussion}\label{results-and-discussion-1}

As in Exp. 1, our main question was which tracing children thought was
better. As predicted, children were significantly more likely to choose
the tracing praised by the Selective Teacher (72.5\%, \(p\) = 0.006,
Binomial Test). To investigate whether children's age predicted their
choice, we fit a logistic regression model:
\texttt{Tracing\ choice\ \textasciitilde{}\ Age\ in\ Months\ +\ (1\ \textbar{}\ Subject)}.
Children's age did not predict their choice of tracing (\(B\) = 0.3,
\(z\) = 0.34, \(p\) = 0.73).

These results suggest that children were more likely to endorse the
praise given by a teacher who had previously provided selective praise
that covaried with the actual quality of the tracings over the praise
given by a teacher who had indiscriminately praised all tracings.

However, when asked which teacher was trying to be nice, children did
not show a preference for either teacher (55\%, \(p\) = 0.64, Binomial
Test); it is possible that children's responses to this exploratory
question were influenced by their answer to the Test Question.

Note however that the patterns of the two teachers' praise differed in
two important ways. First, the feedback (positive versus neutral)
appropriately co-varied with the actual quality of the tracings or it
was indiscriminate; the Selective Teacher provided praise that was
\textit{congruent} with the tracings' quality, whereas the Overpraise
Teacher praised all tracings independent of their quality. Second, the
feedback also differed in the \emph{frequency} of praise; the Selective
Teacher praised only 3 of the 6 tracings while the Overpraise teacher
praised all 6. Thus, it is possible that children have a simple
heuristic that praise from someone who rarely praises is more
informative. We addressed this alternative explanation in Experiment 3.

\section{Experiment 3}\label{experiment-3}

In Experiment 3, we asked whether children distinguish between teachers
who provided positive feedback to the good tracings and neutral feedback
to the bad tracings (Selective Teacher, as in Exp. 1 \& 2) from a
teacher who provided positive feedback to the bad tracings and neutral
feedback to the good tracings (Selective-Incongruent Teacher). Because
the frequency and the overall valence of their feedback was matched, the
critical difference was whether their feedback was congruent or
incongruent with the actual quality of the drawings.

We predicted that if children track whether the content of feedback
appropriately co-varies with the quality of the drawings, they should
trust the evaluation of the Selective Teacher more than the evaluation
of the Selective-Incongruent Teacher. Given the absence of an age trend
in Exp. 2, we limited our recruitment to 4-year-olds.

\subsection{Methods}\label{methods-2}

\subsubsection{Participants}\label{participants-2}

Twenty-three 4-year-olds (15 female, \(M_{Age}\)(SD) = 4.6(0.3), range =
4 - 5) were recruited from a university preschool. An additional 5
subjects were tested but excluded due to failure on the warm-up or
memory check questions.

\subsubsection{Stimuli and Design}\label{stimuli-and-design}

Stimuli were identical to Experiment 2 except the Overpraise Teacher
video was replaced with the Selective-Incongruent Teacher video. The
Selective-Incongruent Teacher was similar to the Selective Teacher but
provided praise in the opposite way: she praised the bad tracings and
gave neutral response to the good tracings.

\subsubsection{Procedure}\label{procedure-2}

The procedure was identical to Experiment 2. As in Experiment 2, the
order of the videos and the actor playing each type of teacher was
counterbalanced.

\subsection{Results and Discussion}\label{results-and-discussion-2}

Our main question was whether children would endorse the praise from the
Selective Teacher, even when the relative frequency of the two teachers'
praise was matched. Indeed, children were more likely to choose the
tracing praised by the Selective Teacher than the one praised by the
Selective-Incongruent Teacher (73.91\%, \(p\) = 0.035, Binomial
Test)..\footnote{As for the exploratory question (``who was trying to be
  nice?''), we did not predict a pattern to emerge as neither teacher
  was trying to be nice; children's responses were not different from
  chance (60.87\%, \(p\) = 0.405, Binomial Test).\} We also fit the same
  logistic regression model as in Exp. 2
  (\texttt{Tracing\ choice\ \textasciitilde{}\ Age\ in\ Months\ +\ (1\ \textbar{}\ Subject)},
  and found that children's age did predict their choice of tracing
  (\(B\) = -6.28, \(z\) = -2.05, \(p\) = 0.04)}

These results suggest that children are not simply responding to the
relative frequency of praise; they are sensitive to whether the teacher
provides feedback that appropriately co-varies with the actual quality
of tracings. Unlike Exp. 2 (and despite the narrower age range), we did
find a small but significant effect of age. Given that there was no
effect of age in Exp. 1, further work is needed to better understand how
this capacity develops throughout the early childhood years.

\section{General Discussion}\label{general-discussion}

Across three experiments, we examined whether adults and children infer
the informativeness of others' praise, and use it to evaluate their own
or others' work when they themselves are unable to judge. Experiment 1
verified that adults reliably distinguish a teacher who selectively
praised from a teacher who indiscriminately overpraised. Experiment 2
provides support for our main hypothesis: Even 4-5 year-old children
distinguish between these two teachers and use their relative
informativeness to evaluate the quality of their own work. Experiment 3
further suggests that children's responses are not driven by the simple
heuristic that people who praise rarely are more informative; when the
frequency and the overall valence of praise were matched, 4-year-olds
still endorsed the praise from the teacher who had previously provided
feedback that was both selective and congruent with reality.

Praise is a useful source of information for learning about our skills
and abilities; determining whose praise to trust or discount is critical
for effectively learning about the self. Prior work has found that
children hold strong beliefs about their competence, predicting that
they will perform better in the future than they are able to (Schneider,
1998) and judging that they have performed well when they have not
(Hembacher \& Ghetti, 2014). We do not know whether this bias about
their own abilities would cloud their evaluation of the informativeness
of feedback directed at them (not another student). Our experiments do
show, however, that they were able to \textit{apply} feedback to make
predictions about their own work -- thus their optimism did not
completely overpower their ability to integrate others' prior feedback
into their assessments of their own work. Even though children
constantly receive praise, they rationally decided whose praise to trust
based on the past history of the evaluator.

These results are consistent with a growing body of literature that
suggests children reason about others' informativeness based on the
information they provide (Gweon \& Asaba, 2017; Sobel \& Kushnir, 2013).
Going beyond using facts about the physical world (e.g., labels of
objects, causal functions of artifacts), children also used information
about the quality of work (e.g., quality of tracings) to decide whether
or not to trust someone's evaluative feedback. More specifically,
children's inferences were based on the statistical dependence between
the pattern of praise and the quality of the work being praised. Thus,
children's early-emerging sensitivity to statistical information (Gweon
\& Schulz, 2011) might also support inferences about others'
informativeness.

Recent work suggests that adults are sensitive to the presence of
competing goals in communicative contexts (e.g., being informative
vs.~being polite, Yoon et al. (2016)). In line with this finding, adults
in our study readily identified which teacher was trying to be nice.
However, this exploratory question about niceness did not yield
informative results from children; it is possible that they did not
distinguish ``trying to be nice'' and ``being nice,''\footnote{An
  earlier pilot with adults suggested that the Selective Teacher could
  be considered (genuinely) nice for being informative.} or perseverated
based on their response to the Test Question. Future work might examine
whether children infer different communicative goals of evaluators from
the statistical patterns of their praise.

Just as evaluators may have competing social goals to be informative or
nice, learners also have reasons to seek informative feedback or warm
compliments from others. Our study emphasized the learner's goal by
explaining that they had to choose a tracing to enter into a contest
(Exp. 1) or show to their teacher (Exp. 2 \& 3). Yet, learners may have
diverse goals in approaching others for feedback; they could want honest
evaluation of their performance, or affirmation of their efforts to
\textit{feel better} about themselves. An open question is whether young
children differentially weight praise based on their own goals, and how
this tendency might change with age. It is possible that younger
children generally seek more affirmation than evaluation, but their goal
might also vary depending on their competence in the domain and the
relative difficulty of the task; for instance, learners might prefer
encouragement when they are struggling on new or difficult tasks. The
kind of feedback children seek might also depend on their relationship
to the evaluator, desiring more affirmation from parents and expecting
more objective evaluations from teachers. Future work might ask how
children actively choose which teacher to approach depending on their
goals (e.g., informativeness versus affirmation).

In the current study, children observed repeated instances of praise
from two teachers whose praise was clearly aligned (Selective Teacher)
or misaligned (Overpraise and Selective-Incongruent Teacher) with the
quality of the tracings. In order to ensure well-controlled presentation
of two teachers who varied only in their pattern of praise, children
watched videos of teachers instead of seeing real teachers praise their
own (or their peers') work. In real life, however, children's
observation of evaluative feedback is often noisier, and unfolds in a
complex social environment where both the evaluators and the students
must navigate multiple competing goals. Therefore, even though these
results are suggestive of an early-emerging sensitivity to the
informativeness of praise, further work is needed to better understand
how such sensitivity might manifest in real-world contexts. Furthermore,
given that our participants were from a university preschool where
children receive ample social support from adults, future work might ask
whether our findings generalize to a broader population of preschoolers
who might experience varying levels of praise, encouragement, and
support.

The ability to make an independent assessment of the quality was
critical for success in our task. Indeed, if you do not know how good a
tracing is, you cannot tell whether a teacher's feedback is appropriate.
In our paradigm, we used clearly good or bad tracings and ensured that
all children could accurately assess their quality. However, we rarely
have absolute certainty about the quality of our own work. When do
children rely on their own assessment to infer others' informativeness,
and when do they rely on others' evaluative feedback to inform their own
assessments? One possibility is that children integrate their own
certainty and others' informativeness to learn about themselves (i.e.,
did I do well?) or about the evaluators (i.e., is this person
informative?). The presence of a majority opinion (Burdett et al., 2016)
or conflicting evaluations (Bridgers, Buchsbaum, Seiver, Griffiths, \&
Gopnik, 2016) may further influence these decisions.

Constructive feedback provides insights into learners' strengths and
weaknesses, and guides their future learning to maximize opportunities
for growth. Our results suggest that the ability to seek constructive
feedback might start early in life. Even for young children, praise is
more than something nice; they track the informativeness of others'
evaluative feedback and use it to infer the quality of their own work.

\section{Acknowledgements}\label{acknowledgements}

We thank Molly Irvin and Habin Shin for help with data collection, and
Athena Braun, Fernanda Kramer, and Johnny Matheou for help in stimuli
creation. We also thank the parents and families of Bing Nursery School.
This work was supported by an NSFGRFP to MA, a grant from the Stanford
Child Health Research Institute to EH, a gift from Kinedu, Inc. to EH
and MCF, and Stanford Psych-Summer funding to SQ and BA.

\vspace{1em}

\fbox{\parbox[b][][c]{7.8cm}{\centering All data and code for these analyses are available at\ \url{https://github.com/masaba/uncertainpraise}}}
\vspace{1em}

\section{References}\label{references}

\setlength{\parindent}{-0.1in} \setlength{\leftskip}{0.125in} \noindent

\hypertarget{refs}{}
\hypertarget{ref-bonawitz2016computational}{}
Bonawitz, E., \& Shafto, P. (2016). Computational models of development,
social influences. \emph{Current Opinion in Behavioral Sciences},
\emph{7}, 95--100.

\hypertarget{ref-boseovski2012trust}{}
Boseovski, J. J. (2012). Trust in testimony about strangers: Young
children prefer reliable informants who make positive attributions.
\emph{Journal of Experimental Child Psychology}, \emph{111}(3),
543--551.

\hypertarget{ref-boseovski2017role}{}
Boseovski, J. J., Marble, K. E., \& Hughes, C. (2017). Role of
expertise, consensus, and informational valence in children's
performance judgments. \emph{Social Development}, \emph{26}(3),
445--465.

\hypertarget{ref-Bridgers:2016ge}{}
Bridgers, S., Buchsbaum, D., Seiver, E., Griffiths, T. L., \& Gopnik, A.
(2016). Children's causal inferences from conflicting testimony and
observations. \emph{Developmental Psychology}, \emph{52}(1), 9--18.

\hypertarget{ref-brophy1981teacher}{}
Brophy, J. (1981). Teacher praise: A functional analysis. \emph{Review
of Educational Research}, \emph{51}(1), 5--32.

\hypertarget{ref-brummelman2014s}{}
Brummelman, E., Thomaes, S., Orobio de Castro, B., Overbeek, G., \&
Bushman, B. J. (2014). ``That's not just beautiful -- that's incredibly
beautiful!'' the adverse impact of inflated praise on children with low
self-esteem. \emph{Psychological Science}, \emph{25}(3), 728--735.

\hypertarget{ref-Burdett:2016es}{}
Burdett, E. R. R., Lucas, A. J., Buchsbaum, D., McGuigan, N., Wood, L.
A., \& Whiten, A. (2016). Do Children Copy an Expert or a Majority?
Examining Selective Learning in Instrumental and Normative Contexts.
\emph{PLoS ONE}, \emph{11}(10), e0164698.

\hypertarget{ref-delin1994praise}{}
Delin, C. R., \& Baumeister, R. F. (1994). Praise: More than just social
reinforcement. \emph{Journal for the Theory of Social Behaviour},
\emph{24}(3), 219--241.

\hypertarget{ref-gweon2017order}{}
Gweon, H., \& Asaba, M. (2017). Order matters: Children's evaluation of
underinformative teachers depends on context. \emph{Child Development}.

\hypertarget{ref-gweon201116}{}
Gweon, H., \& Schulz, L. (2011). 16-month-olds rationally infer causes
of failed actions. \emph{Science}, \emph{332}(6037), 1524--1524.

\hypertarget{ref-gweon2014sins}{}
Gweon, H., Pelton, H., Konopka, J. A., \& Schulz, L. E. (2014). Sins of
omission: Children selectively explore when teachers are
under-informative. \emph{Cognition}, \emph{132}(3), 335--341.

\hypertarget{ref-hembacher2014don}{}
Hembacher, E., \& Ghetti, S. (2014). Don't look at my answer: Subjective
uncertainty underlies preschoolers' exclusion of their least accurate
memories. \emph{Psychological Science}, \emph{25}(9), 1768--1776.

\hypertarget{ref-henderlong2007effects}{}
Henderlong Corpus, J., \& Lepper, M. R. (2007). The effects of person
versus performance praise on children's motivation: Gender and age as
moderating factors. \emph{Educational Psychology}, \emph{27}(4),
487--508.

\hypertarget{ref-henderlong2002effects}{}
Henderlong, J., \& Lepper, M. R. (2002). The effects of praise on
children's intrinsic motivation: A review and synthesis.
\emph{Psychological Bulletin}, \emph{128}(5), 774.

\hypertarget{ref-kanouse1981semantics}{}
Kanouse, D. E., Gumpert, P., \& Canavan-Gumpert, D. (1981). The
semantics of praise. \emph{New Directions in Attribution Research},
\emph{3}, 97--115.

\hypertarget{ref-Koenig:2005p146}{}
Koenig, M. A., \& Harris, P. L. (2005). Preschoolers mistrust ignorant
and inaccurate speakers. \emph{Child Development}, \emph{76}(6),
1261--1277.

\hypertarget{ref-lane2013informants}{}
Lane, J. D., Wellman, H. M., \& Gelman, S. A. (2013). Informants' traits
weigh heavily in young children's trust in testimony and in their
epistemic inferences. \emph{Child Development}, \emph{84}(4),
1253--1268.

\hypertarget{ref-mueller1998praise}{}
Mueller, C. M., \& Dweck, C. S. (1998). Praise for intelligence can
undermine children's motivation and performance. \emph{Journal of
Personality and Social Psychology}, \emph{75}(1), 33.

\hypertarget{ref-ronfard2017preschoolers}{}
Ronfard, S., \& Lane, J. D. (2017). Preschoolers continually adjust
their epistemic trust based on an informant's ongoing accuracy.
\emph{Child Development}.

\hypertarget{ref-schneider1998performance}{}
Schneider, W. (1998). Performance prediction in young children: Effects
of skill, metacognition and wishful thinking. \emph{Developmental
Science}, \emph{1}(2), 291--297.

\hypertarget{ref-seiver2013did}{}
Seiver, E., Gopnik, A., \& Goodman, N. D. (2013). Did she jump because
she was the big sister or because the trampoline was safe? Causal
inference and the development of social attribution. \emph{Child
Development}, \emph{84}(2), 443--454.

\hypertarget{ref-sobel2013knowledge}{}
Sobel, D. M., \& Kushnir, T. (2013). Knowledge matters: How children
evaluate the reliability of testimony as a process of rational
inference. \emph{Psychological Review}, \emph{120}(4), 779.

\hypertarget{ref-yoon2016talking}{}
Yoon, E. J., Tessler, M. H., Goodman, N. D., \& Frank, M. C. (2016).
Talking with tact: Polite language as a balance between kindness and
informativity. In \emph{Proceedings of the 38th annual conference of the
cognitive science society}. Cognitive Science Society.

\end{document}
